% !Mode:: "TeX:UTF-8" 

\BiChapter{研究可编程设备加速主机侧网络方法}{NIC}


\BiSection{本章引论}{}

\BiSection{问题背景}{aa}

%有什么东西软件做不好的,
%
%为什么要选流量工程 和 压缩加速。一定要好好想明白。 每包计算是最费劲儿的。
%流量系统背景和测量背景, %1
%
%软件流量捕捉 回放能力,

随着数据中心服务器网络接口容量快速增长,处理与网络数据相关的服务已越来越耗费主机CPU的计算资源。主机虚拟化、流量工程、网络监管等功能在现代化网络管理中已经占据重要的位置。尽管目前发展出一系列RDMA、DPDK等基于网卡的网络数据包快速搬运架构,但它们只针对于块儿数据的传递进行了优化,对于需要“每包处理”的任务依然没有良好的对策。



本文将复杂网络计算问题分为两类任务类型,一种类型是由于数据包到达频繁而触发的大量计算,也许这类型计算并不复杂(处理器使用几条指令即可完成计算),但由于需要对每个数据包都进行处理从而导致计算量庞大、CPU无法胜任,例如查找、转发、分类等。第二种类型是由于计算过程复杂(处理器需要耗费多条指令才可完成计算),导致CPU无法提升针对每个数据包处理的速度,进而造成网络分发数据包个数降低、性能需求无法胜任,例如防火墙、安全分析等。数据中心运营商面对着不断增长的功能需求和性能需求,同时也面对着需要降低运营成本提升能源利用率和绿色环保。

网络随路计算(in-network computing)是一种解决网络程序性能差的有效途径。网络随路计算是指在网络数据传输链路中增加特定功能的硬件设备,使得数据包在传送到主机内部之前,就完成了网络任务中的相关计算需求。当计算从主机内,下方到了网络内后,这便释放了软件处理瓶颈,可以节约数据中心内宝贵的CPU处理资源。但主机内的计算任务并不能够全都被硬件卸载,因而我们需要选取可编程数据平面来完成这件事。前文提到过三类可编程网卡他们分别是:基于可编程ASIC的智能网卡,基于NPU的智能网卡,以及基于FPGA的智能网卡。首先本文分析基于ASIC的智能网卡是非“图灵完全”的可编程硬件,尽管其处理性能优异,但无法支持灵活配置,因而本文不考虑。其次基于NPU的智能网卡,NPU由众核处理器架构组成,可提升处理并行度,但这是基于批处理的计算模型。针对目前流式计算和有状态计算,因NPU每核处理性能低,从而导致NPU方案整体运行性能差。

本文从上述两类问题中各选取了一个应用场景,来说明使用基于FPGA的智能网卡具备强大的“每包处理”能力,以及强大的“复杂计算”能力。最重要的是FPGA可以支持网络内流式计算模型,本文在后面提出了一种在FPGA针对流式计算需求的DC抽象方法,可以将有前后状态依赖的“复杂计算”任务卸载到基于FPGA的可编程硬件。

%The Case For In-Network Computing On Demand 分类举例说明



\BiSection{系统架构}{aa} %1

\BiSubsection{软件向硬件卸载分析}{aa}

CPU通过循环取指令等操作,完成通用的可计算任务。计算的数据通常有前后依赖关系,CPU对于此类计算效率较低:由于中间计算状态在计算完成之后必须放回数据存储区,而下次重新拿回状态数据又需要再次搬运,数据在计算核心与存储池之间多次往返对计算最终结果是无意义操作。而即使使用众核处理器也无法优化此过程,虽然众核处理器可并行计算多个任务,但同一时刻每核心与其他核心处理内容并无关联。对于前后有依赖关系的处理,并行并不能加速其中一组数据的处理进程。在网络领域亦是如此,网络数据包到达密度很大,留给每个包的处理时间很有限,然而通常一个CPU无法在如此短时间内真正处理完一个包的触发计算。


1)软件处理延迟大。

目前高速网络处理器可以设计为核间流水线模式,每个核只处理固定的一步计算。当一个数据包触发的计算包括多个前后依赖计算时,人们使用多个核串行处理这组计算。即每个核心领取一个固定的快速处理任务,这样每个核都可以以最快的速度处理完当前步骤,然后交给后续核心继续处理。这样数据包的处理吞吐其实就可以达到某一个核心的最大速度,而处理延迟则是这些核心之间传递完整一次的时间。不难发现,数据在众核之间搬运会遇到访存时间过长、访存请求冲突等现象。虽然使用CPU可以获取很大的灵活性,但这种方式进一步增大了每个数据包处理的时延。如今在云端加速AI计算的场景下,高频次小包、数据快速到达的需求越来越高,这使得以CPU处理网络数据包造成很大的云计算性能瓶颈。后续我们通过本文所举的例子可以明显提现这一点。

2)软件处理时间精度低。

同时,由于共享内存与总线架构,数据在CPU核心、存储池之间搬运会造成随机的请求冲突、甚至计算等待,这将中断顺序串行处理节拍。由于这条处理链中的处理速度会受到这条处理链最慢处的制约,因而处理链的性能表现总以最慢的瓶颈向外表现。
软件的处理收到操作系统的指挥,操作系统一般会划分时间片区来分配给每一个待处理的任务。虚拟化的操作系统内有很多任务进程,操作系统会划分很多时间片区。这也进一步降低了CPU的专用任务处理性能,还带来处理时间精度不足的现象。后续我们通过本文所举的例子可以明显提现这一点。

3)软件功能向硬件卸载。

卸载是指将工作负载从一个实体转移到另外一个实体上去。软件功能的卸载遵循一定原则:

\begin{itemize}
	\item 新实体可以做与原来实体相同的工作,但资源耗费更少。
	\item 新实体比另外一个实体更快地完成某种类型的工作。
	\item 新实体有能力做除此外的其他任务。
\end{itemize}

硬件功能卸载已经广泛使用在计算机领域。早期,开发人已经把音视频处理从CPU完全卸载到图形处理器和音频芯片中。此外还有一些例子:数据包校验计算卸载到网卡,传输大宗数据时使用DMA控制器,在硬盘控制器中做数据错误恢复。事实上,从CPU中卸载出来的功能都比单纯由CPU执行能源效率高很多。但本文并不是强调CPU无用论,因为CPU是整个服务器中最具有灵活性的器件。卸载任务应该是简单、机械的,CPU是系统内可编程、可运行花样繁多的算法,从而具有很高的难以替代性。

卸载通常是一种优化技术,与其他优化方法一样,仅在合适场景下适用,虽然他增强了计算效率,但这种卸载方式并不是完美的:

\begin{itemize}
	\item 增加了软硬开发人员的工程任务量。
	\item 增加新的硬件,使系统结构复杂。
	\item 设计制造新硬件,增加研发周期、增加使用成本。
	\item 由于大量装配特殊器件,系统可移植性降低。
\end{itemize}

对于灵活多变的网络流量的处理任务,大多为分步骤且数据包信息前后依赖关系高。前文提到,如果使用基于NPU的可编程网卡,虽然灵活性可以被满足,但依然存在性能问题(单核性能低,单个数据包处理延迟大)。使用基于可编程硬件(FPGA)的智能网卡卸载网络功能,可以有效解决上述问题。


\begin{table}[!ht]
	\renewcommand{\arraystretch}{1.2}
	\centering\wuhao
	\caption{FPGA的优势与劣势} \label{table:FPGAstrengthandweakness} \vspace{2mm}
	\begin{tabularx}{\textwidth}{*{2}Y}
		\toprule[1.5pt]
		FPGA所擅长的处理内容 & 不适宜使用FPGA处理的任务 \\
		\midrule[1pt]
		流式数据通路(视频处理) & 多决策算法 \\
		并行算法 & 递归 \\
		低延迟、确定延迟系统 & 依赖于其他大型软件库 \\
		与硬件接口紧邻 & 浮点数学计算 \\
		\bottomrule[1.5pt]
	\end{tabularx}
\end{table}

表\ref{table:FPGAstrengthandweakness}展示了使用FPGA作为卸载功能目的实体与CPU和ASIC对比的相对优劣点。虽然FPGA完全有能力完成分支多且复杂的决策算法,但相比于CPU,在FPGA上实现会付出更多编码时间,不如使用通用处理器经济收益好。另外相对软件而言,FPGA的处理过程具有精确无抖动(jitter-free)的时序效果。虽然独特优化的软件代码可以近似达到这种效果,但它依然严重地受到调度器,中断,不同处理路径的影响而产生抖动。


\BiSubsection{软件算法的硬件抽象方法}{aa}

在高性能的大数据领域,批处理技术已经可以满足非常多的使用场景,但随着数据时效性价值提升,一种流式计算概念被提出。流式计算是实时的,常驻的,与Hadoop,MaxCompute,Spark等系统不同(它们是离线批处理),流式计算对延迟属性要求更为苛刻。而目前在大数据和高性能网络中的处理就更需要考虑如何利用FPGA来加速流式计算的需求。
%阿里巴巴 大数据page70 流式计算的特点
\begin{figure}[!ht]
	\centering
	\includegraphics[scale=1]{fpgaprogramabstraction.pdf}
	\caption{FPGA硬件编程抽象} \label{fig:fpgaprogramabstraction}
\end{figure}

1)FPGA编程抽象。
		
基于FPGA的可编程逻辑,可以完成很多种计算。在软件思维转向硬件化的时候,如图\ref{fig:fpgaprogramabstraction}所示,给出了三种并行化加速方法:同一种功能($F(x)$)的复制展开并行;一组数学计算的内部隐含(四则运算分配律和结合律)逻辑并行;分支选择的选路并行。



2)网络卸载的数据---计算DC抽象方法。

\begin{figure}[!ht]
	\centering
	\includegraphics[scale=1]{dcprinciple.pdf}
	\caption{FPGA加速流式计算的DC抽象方法与原则} \label{fig:dcprinciple}
\end{figure}

网络计算卸载到FPGA中,需要在FPGA的并行计算加速基础上,引入一种适合于流式计算FPGA的编程抽象,本文称其为“数据---计算”模型(Data--Computing,DC抽象)。如图\ref{fig:dcprinciple}所示,每个节点内都包括一个或一系列功能(函数$F(x), G(x)$等),本文称其为“计算”,在不同的计算节点之间是数据流(以数据包为驱动),本文称其为“数据”。如何排列和组织这种“计算和数据”是DC抽象的核心内容。一般数字电路会设计为时序逻辑,也就是以时钟周期时间跨度为节拍的计算。各个逻辑块需要在同一个节拍内计算出相应结果。对于在数据处理时前后无关的计算可以归到内部并行中去,所以流式数据处理方法需要挑出数据计算时有前后相关的部分,并把它们按照一定拓扑串联起来。DC抽象基于上述分析提出在FPGA内处理网络流式计算的加速的原则:

\begin{itemize}
	\item 在流水线的每一级具有确定的节拍数。
	\item 数据前后内容具有依赖性。%流式处理要分层,只把前后数据相关的分层,
	\item 有固定的流水线拓扑,但是数据在拓扑间可任意流动。
\end{itemize}

下面将重点介绍本文如何利用DC抽象原则,对基于可编程硬件的网络高时效性网络任务和高密度计算任务卸载加速的研究。

%上面的部分再画个图吧。。。。。。

\BiSection{网络测量---网络流量捕获与回放}{aa} %4

\BiSubsection{概述}{aa}

网络测量和分析是监控、调试和管理网络,进而设计反馈和优化网络结构以及改善网络服务质量的重要手段。近年来,互联网与网络应用的不断发展,网络内服务器遭遇的各类破坏与攻击也随之与日俱增,网络安全问题日益手段人们的重视。因此,在大型数据中心内,对入侵攻击的检测与防范、保障信息系统和网络系统安全已经成为研究人员的共识。为了快速确定网络攻击的发生,如何设计一种可以实时检测网络流量的网关系统变得异常重要。例如,当网关系统发现大规模DDoS攻击时,可以主动做出丢包等拦截动作,但是对正常数据包应采取放行策略。

这类系统对低传输时延、高吞吐迫切需求。对于复杂网络协议、和内容的分析,一般需要使用大型服务器的多核CPU。由于网络中流的数目很多,因此可以将不同的流量分配到不同的并行CPU中去做并行入侵检测。但对极高流速的网络数据包的捕获,分发等工作却面临很大的性能压力。经过检测系统处理完的数据包还需按先后顺序汇聚,保证多核并行系统不会对原始流量数据包的内在逻辑产生影响。因其特殊的处理需求,以及数据包高吞吐特性,因此安全监测系统的网络收发部分是一项极为重要的挑战。因此本文的主要研究点服务器端网络数据的接入、分发与回流,其余的内容检测等属于上层软件应用,不在本文讨论范围之内。





\BiSubsection{问题分析}{aa}

对于流量接入而言,其挑战在于如何快速地捕获足够的网络流量。经典的如TCPDump以及TCPTrace等基于软件的流量捕获工具,可以完成将某网络接口数据包镜像并缓存的功能。简单来说数据包首先到网卡,然后通过软中断的形式触发内核对其处理,通过DMA写入内存,最后再被应用程序读取。软件首先使数据包经过BPF(伯克利包过滤,BPF能够通过比较第2、3、4层协议中各个数据字段值的方法对流量进行过滤)包过滤器,通过过滤的数据包放入bufferQ缓存(给应用程序读取数据包的缓冲队列),tcpdump就是从bufferQ中抓取数据包。不难发现,这类操作必须由内核态到用户态两次传输,一个数据包需要经过多次CPU处理循环。

其主要问题有两点:操作系统时间精度低,软件处理丢包严重性能低下。首先,时间精度低。为了记录数据包接入的速度特征分布,我们需要对每个数据包首增加到达时间戳,这样在令数据包回流网络使可最大限度的保证不影响对每个流的网络服务特性。数据包到达后发送软中断,软件为其添加时间戳时已经有了系统误差。这种误差来自于操作系统时间管理与任务轮询的不精确性(例如图\ref{fig:jitter})。这样的不精确会给流分布的记录造成误差,最终导致某一段数据流抖动(jitter)增大,改变数据流特性。加之在真实场景中,软件收发包通过虚拟层,协议栈,内核态,用户态等多步骤处理,累计的总体时间抖动很大。第二,性能低下。目前在linux服务器环境下的捕获的丢包率偏高(例如图\ref{fig:tcpdumploss}),与真实网络接口速度差距过大。由于需要对串行数据中每个数据包依次增加时间戳,因而在此过程中无法引入软件并行,从而仅依靠单核极易造成性能瓶颈。



\begin{figure}[htbp]
	\centering
	\begin{minipage}[t]{0.48\textwidth}
		\centering
		\includegraphics[scale=1]{jitter.pdf}
		\caption{软件环境网络协议栈发包时间精度} \label{fig:jitter}
	\end{minipage}
	\begin{minipage}[t]{0.48\textwidth}
		\centering
		\includegraphics[scale=1]{tcpdumploss.pdf}
		\caption{TCPdump收包工具丢包率} \label{fig:tcpdumploss}
	\end{minipage}
\end{figure}

流量回放技术基于流量捕获基础之上完成,除前文提到用于实时入侵系统外,在离线性能测试、模拟攻击等领域还有重要作用。人工构造的流量并不能完全重现真实环境中的网络,研究人员利用流量回放还能用以测试实时网络安全系统。

为应对如今超大规模接口带宽,基于软件的流量捕获和回放系统遇到了极大的可扩展性问题,已无法满足目前的行业需求。本文以解决此问题为目标,在主机端使用可编程硬件网卡技术,来大幅度提升服务器捕获回放数据包性能。



\BiSubsection{设计}{aa} %总述,本章将实现,,,,然后,分两部分写。

网络流量的捕获、统计与回放是统一的一套体系,在软件层面使用使用多核间的流水线完成加速。在流水线中的所有数据包,都可以看做同时受到各个前后功能的处理,这样可以大幅增加吞吐。如图\ref{fig:caprepsoftway},本文的基本思路是将软件层的所有瓶颈功能全部下放到可编程硬件,只留轻负载的文件存储管理系统。如图\ref{fig:caprephardway}所示

\begin{figure}[htbp]
	\centering
	\begin{minipage}[t]{0.48\textwidth}
		\centering
		\includegraphics[scale=1]{caprepsoftway.pdf}
		\caption{基于软件的设计方案} \label{fig:caprepsoftway}
	\end{minipage}
	\begin{minipage}[t]{0.48\textwidth}
		\centering
		\includegraphics[scale=1]{caprephardway.pdf}
		\caption{基于可编程智能网卡的设计方案} \label{fig:caprephardway}
	\end{minipage}
\end{figure}

\BiSubsection{优化}{aa}

向100G出发


\BiSection{统计---网络测量实时压缩}{aa} %4

\BiSubsection{设计}{aa}

1)压缩算法

2)压缩算法硬件实现

3)效果与问题



\BiSubsection{优化}{aa}

1)符合DC抽象的算法优化方法

2)无偏估计证明




\BiSection{软硬一体化的系统实验平台}{aa} %2

\BiSubsection{软件}{aa}

\BiSubsection{硬件}{aa}

\BiSection{性能评估}{aa}

时间精度

吞吐

估算

cacti比较











在主机端,网络与计算的需求,

在第二章中提到NPU的可编程性最强且可以使用软件控制,由于他单核性能问题,尽管需要使程序员更换硬件思维,但业界普遍已经采取FPGA的方式来加速网络计算。

本章的主体是网络功能使用可编程硬件加速。

主机端网络开销最大的还属流量工程和网络内计算,因为每包处理,目前百G NIC 包速率已经达到150Mpps,对于每包操作的计算任务,CPU负载很大

1)分析软件到硬件卸载的可行性, 背景
	
	软件可扩展分析,



	

%流式处理《阿里巴巴大数据》page70


















