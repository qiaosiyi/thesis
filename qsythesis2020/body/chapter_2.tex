% !Mode:: "TeX:UTF-8" 

\BiChapter{相关工作综述}{pdpintro}
\label{sec:pdpintro}



%重要概念介绍

%重要技术的介绍

%研究的基石

%研究的关键问题



%目标,软件功能硬件卸载,提速。在硬件中增加可编程逻辑的对外性能。


\BiSection{本章引论}{}
本章综述了国内外网络基础设施技术的演进,主要分析其主要技术特征和局限短板,重点关注了现阶段实际情况下SDN可编程数据平面的灵活性与性能矛盾点,为本文研究工作指明了方向和意义所在。







\BiSection{网络可编程的发展历程}{}
%研究领域发展趋势介绍





\BiSubsection{软件实现---早期网络基础设施}{} %IP10k https://mp.weixin.qq.com/s/1tUXilmvbIzlMoDQUuC2Jg 《可编程数据平面调研_说的还不错.pdf》

%主旨:过去软件的好处,软件的坏处,现在软件的好处,软件的坏处

网络对于业务的基本价值是网络实现了数据在计算机之间的任意传输。在早期\footnote{上世纪90年代中期以前},由于用户数量、计算机算力、存储、硬件性能都过于微弱,作为连接所有终端、服务与用户的管道,网络的主要特点集中在连通性、可行性和初期探索性上。在一个简单的星型拓扑中,一个路由器其实就是一台普通计算机。在学术和产业界的初期,人们并没有意识到网络需要单独拎出使其成为一套独立系统的价值。这在侧面也体现出软件作为网络实施载体的特点:“灵活性”。即:对于处理并实现一个新兴事物,软件可以发挥其巨大的灵活性优势,使其可以作为一种为数不多的手段,快速实现工程师学者的任意的新的思想。

后期随着社会生活、技术进步,步入信息时代之后逐渐发现人与人之间数字信息交互的需求和价值越来越大。因而研究重点开始关注在如何实现快速的包交换、路由查找。为此人们开始提出各种快速交换的数据结构:Cache优化、哈希表、Radix Tree(树查找)等。很长时间基于软件的转发设备核心架构都没有变化,唯一变化的是跟随摩尔定律成长的芯片技术。CPU和存储每18月性能翻番,网络设备的性能也顺势而上,人们对网络的发展信心十足。网络处理从单CPU向多CPU并行,向分布式存储cache结构进行了小小扩展,但也好像失去了创新的动力。然而人们对信息量需求的增长却大大快于摩尔定律。到2011年底,我国互联网入户带宽平均接近20Mbps\citeup{20112m}。从最初14.4Kb的拨号上网,网络容量的发展几乎是以每18个月翻10倍的速度在增长。在数百兆的路由性能要求下用软件作为转发设备基础比较合适,但如果核心网要升级到1G或数十G以上更高的带宽就会面临技术、成本等多方面的瓶颈。

\BiSubsection{向硬件过渡}{} %IP10k

数据包交换对于CPU来讲是一种很累的工作。虽然数据包转发算法既简单、又高效,但面对无穷无尽的任务量,依靠指令集的软件转发架构存在访存效率差、CPU无法批处理等劣势。这时研究人员抛弃了基于指令集的软件架构,开始思考基于专用硬件电路(Application-specific integrated circuit, ASIC)的数据包处理模型。此时硬件转发的发展目标是如何增大交换设备的交换容量、以及研究具有更好的可扩展性的设计方案。电路交换Crossbar(交叉开关)[1]架构追求$N$队列输入到$N$队列输出的无阻碍转发,其思想的本质是使用一种二维电子开关(Switching)矩阵来增强交换设备的转发能力。矩阵中有$N^2$个开关交点,可以实现任意的$N_i$输入映射到$N_j$输出,也易实现多对一、一对多映射。由于开关数量众多控制器硬件算法难度高[2],以及传输冲突等问题[3],此后的一系列技术创新集中在如何降低Crossbar的管理时延、提高理论吞吐容量[4,5]。人们也在思索如何在扩展交换容量时节约芯片面积,其中重要的思想是由单模块交叉开关联结为多交叉开关结成的网(fabric)。有专用硬件电路的加持,业界把单芯片交换能力提升至12.8Tbps[7]。能够支持在一个大规模数据中心内可以支持128台配置有100G网的卡服务器形成一个小区进行高速互联,这样的组网计算机的并行处理能力已经足够一个通常规模大数据算法使用。单芯片容量升高会使晶体管面积成$O(N^2)$规模增长从而变得不再划算。如果想支持512台服务器,网络架构商可以选用两级Spine\&Leaf(骨干与边缘)架构,使用12\footnote{12=4(Spine)+8(leaf),每个leaf节点对外暴露一半的接口容量,最终扩展4倍到达51.2Tbps}块12.8Tbps的交换芯片组成一个51.2Tbps的扩展规模网络。

当设备被大规模组网时,网络的管理问题变得尖锐。早期互联网的发展非常迅速,因为设备的扩展就是简单对接,每个设备独立控制自己,具备对外扩展的策略。随着时间的流逝,网络内产生了成百上千个新IETF RFC(网络工程备忘录)和IEEE标准。设备制造商需要用同一个产品向各类运营商提供服务,这导致在同一款路由设备产品中堆叠的功能特性也越来越多。一些ISP路由设备的源代码甚至超过1亿行,是最复杂电话交换机的10倍以上,要知道电话交换机也曾需要支持上百种协议\citeup{ethane},即使大多数客户只需要其中某一种功能。互联网也为这种高复杂度付出了代价:设备臃肿部件数量庞大、不节能效率低下、价格昂贵、API的设计随意。由于路由设备行业门槛较高,初创企业难以进入市场并发挥创新能力。此时大的路由器供应商也为路由器的可靠性、高复杂性、安全性等问题苦恼,网络的创新速度又变慢了。

%TODO:网络管理是传统的,过去小规模是有优势的,但当规模大了之后收敛慢,编程复杂,可以参考ethan论文 -> From Ethane to SDN and Beyond,然后主要找传统网络的劣势



[1] paper: Fast Switched Backplan for a Gigabit Switched Router.
[2] 变长报文 60\%资源
[3] HOL blocking
[4] 固定 cell based switch 
[5] VOQ vitrual output queueing
[6] CLOS
[7] broadcom switching chip

\BiSubsection{软件定义网络演进---软、硬任务划分,物理隔离}{} %http://blog.sina.com.cn/s/blog_13743c4140102vh7e.html openflow 标准演进过程 
%《软件定义网络 SDN 数据平面带状态转发》 page14
%《阿里巴巴)page10 12

%先说问题,

%再说SDN的解法
1)数据平面的统一化与精简控制软件。

软件定义网络(Software Defined Networking,SDN)\citeup{mckeown2008openflow}的概念赋予运营商集中式或半集中式程序控制的便利。网络设备控制面和数据面的物理隔离,给这种体系架构带来经济学层面的优势:能将复杂的数据平面管理功能软件集中在少数几个地方,具有统一设计的数据平面抽象。最开始人们发现,每一个运行在网络系统里的数以千计的交换机和路由器都运行着一个程序处理器。大量这种分布式控制设备的数据平面运行的软件其实是一样的,但却需要设备数量十分之一\citeup{casado2007ethane}的网络管理员去不停地确保网络正常运转。相比于运维效率低,不确定性才是最危险的。由于传统网络的控制平面是分布式的,在正常运行状态下没有人能够有一个清新的网络运行图,因而在网络出现问题时管理人员很难调试。对于数据平面的设计思想也很直接,数据平面必然完全接受控制平面的控制策略,而数据平面输入输出都是数据报文,那么数据平面内的所有操作都可以由Matching--Action(匹配---执行)模型抽象出来。网络的功能就由远端控制平面上的软件来定义,这将有助于网络的“创新力”。因为网络功能、协议的定义不再只能由设备供应商提供,而能够由真正维护和使用网络的操作人员现场定义/修改。同时,操作人员能够拥有网络全局视图,对保障网络运行和安全控制也有极大的促进。

数据平面与控制平面之间的交互称为“南向接口”,目前南向接口的事实标准是2008年由斯坦福大学提出的OpenFlow协议。OpenFlow协议由最开始的OpenFlow1.0,快速发展到现在的OpenFlow1.6。几年时间,OpenFlow协议已经逐步完善到各个细节应用:流量调度[1],光适配\citeup{openflowoptical},广域网[3],超转发\footnote{Super Packet Transport Network,~SPTN。一种硬件功能组件可分解的高效可编程网络框架。}\citeup{openflowsptn}等。

2)云和虚拟交换。
%ip1w

随着云计算的持续发力,虚拟化成为其中重要技术。虚拟机内部互相通信需求增高,基于SDN的OpenVSwitch同样也令虚拟交换机编程更容易、转发更方便。软件定义网络加速云虚拟化的创新,软件定义网络能够提供非常复杂的虚拟网络语义,支持快速迭代。数据中心网络性能的提升需求远远快于CPU的处理能力的增长,通常来讲,CPU一个核心能够支持10Gpbs的转发性能。对于未来数据中心服务器百G带宽需求,也许需要消耗CPU总体性能的20\%\footnote{以常见Intel志强48核心CPU处理器为例。}。

[1]dynamic flow scheduling for data center networks 和 elastic tree saving energy in data center networks
[2]openflowoptical
[3]
[4]


\BiSubsection{协议无关数据平面可编程演进---可编程性层次划分,逻辑隔离}{} %《可编程数据平面调研_说的还不错.pdf》

%《软件定义网络关键技术及相关问题的研究》 page10


%然而如今众多快速出现的功能,并不是一个固定域就能分类完成。
1)扩充报文编码与设备快速更新

虽然SDN已经将数据平面高度抽象,操作人员可以灵活的定义什么样的流,以及对这种流进行怎样的操作。但是在数据平面内数据包头的匹配域却是预先规划好的。固有转发平面的设计思想会引起如下两个问题:其一,添加新特性需要跟业界讨论、以及等待很长的设备研发时间;其二,在数据平面内固化现实中可能出现的每一个网络协议字段造成宝贵计算资源的巨大浪费。满足新阶段的网络创新需要比SDN更好的灵活性、动态性。因此斯坦福大学提出了P4\citeup{p4}编程语言框架,这种语言有能力重新定义数据平面的包解析模式。P4源代码通过前端编译器编译为中间表示层代码,这个编译过程将提出源代码中的语义逻辑。之后需要根据不同的目标器件再进行后端编译,这个过程最终会生成目标器件对应的机器码,硬件可直接读取。目前P4的目标设备已经有基于ASIC的交换芯片、CPU、GPU和FPGA等多种实现。

P4是与流表式编程不同,它是另外一种维度的高层次可编程概念。在P4框架中,网络操作者可以根据新的设计,创造性地自行设计一种结构的数据包头字段。通过P4源代码,将新的包头结构编译到数据平面形成新的指令。这就实现了灵活定义数据平面解析过程。
P4的目标是让已经部署的硬件网络设备数据平面实现软件定义升级,可以达到在线无插拔地更换新设备的效果。P4的出现也首次实现了数据平面不同逻辑层面上的可编程性。

2)可编程硬件的未来

数据平面可编程概念引发了众多新技术和为解决不同问题所提出的创新实践,如图\ref{pdphistory},本文从不同方向架构梳理这些工作。

\begin{figure}[!ht]
	\centering
	\includegraphics[scale=1]{pdphistory.pdf}
	\caption{可编程数据平面各界发展历史} \label{pdphistory}
\end{figure}

当设备处理接收进来的每个数据包时,数据平面是网络当中最关键的环节。通常需要用到专用的硬件设施,或者经过复杂优化后的软件加速方案。在硬件方面,数据平面可以在ASIC\citeup{de2009plug,anwer2010switchblade,intelflexpipe,rmt,tofino,tofino2},FPGA\citeup{naous2008implementing,yabe2011openflow,netfpga2014,han2015blueswitch,li2016clicknp,wang2017p4fpga,sdnet,firestone2018azure}, 网络处理器\citeup{intelixp4xx,xpliant,netronome},外挂有三态内容地址查找器件(TCAM)的系统\citeup{pagiamtzis2006content},在软件方面有基于快速包分类算法\citeup{feldman2000tradeoffs,kogan2014sax,srinivasan1999packet}的在CPU系统上实现\citeup{greenhalgh2009flow,lincswitch,indigo,cpqd,snabbswitch,ovs,molnar2016dataplane,pisces,panda2016netbricks,p42018behavioral,sonic,dalton2018andromeda}。从2008年提出软件定义网络,由于真实环境性能的需求,业界从未间断地开发基于硬件的可编程数据平面。在P4概念提出来之前,也有类似于半P4的混合型可编程数据平面,受限于设计架构,他们对于短长度域可以实现任意匹配,基本可实现常见协议的数据平面编程,但不能够有效支持宽域\citeup{de2009plug}。

由于硬件可编程技术的加持,外加比虚拟机更轻量级的容器、高速分布式存储、无服务架构、AI对I/O响应速度的要求,使得网络体系架构设计发展繁荣、爆炸增加。相信在未来业界将会出现更多的应用场景,这些场景也将会不断促生更高的网络可编程性、已经更强大的性能。




\BiSection{网络可编程性的“图灵完备”}{}
%完备性与硬件平台的设计思路息息相关。https://www.zhihu.com/question/20115374 知乎问题解答



\BiSubsection{通用可编程性和可编程网卡}{}%虽说FPGA足够灵活,但卸载东西还是有难度
%NP



\BiSubsection{领域内可编程性和可编程转发设备}{}%ipad笔记本






\BiSubsection{可编程数据平面的应用与问题}{}%已经有的成果和应用,展望我们工作的未来
% https://rg0now.github.io/prog_dataplane_reading_list/README.html#org37fc8b1 有很多应用和分类 %《可编程数据平面调研_说的还不错.pdf》
%《阿里巴巴》page11

%《软件定义网络带状态转发》 page22 openflow的不足




\BiSection{网络资源优化}{} %已经有的成果和应用,展望我们工作的未来 周亚东 冷峻园的安全论文

\BiSubsection{软件定义网络安全通道机制}{}
%《软件定义网络关键技术及相关问题的研究》 page11

控制器的信令经由安全通道发往转发设备,实现控制器对数据平面的灵活控制。




\BiSubsection{数据平面流表资源与问题}{}
%ip1w 2.2.1



\BiSection{本章小结}{}
核心思想是在不同时段,人们根据不同的现有即使做取舍从而满足技术需求。











































