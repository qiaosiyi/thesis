% !Mode:: "TeX:UTF-8" 

\BiAppendixChapter{致\quad 谢}{Acknowledgements}


不积跬步无以至千里,回顾读博士7年的时间,我于专业领域的知识与技术都获得了不小的进步。几千天在人生中不长,但也不短,一夜夜的点灯熬油、一行行的实验代码以及一篇篇的文章专著都刻在我人生中,留下了难忘记忆。


首先感谢我的母校西安交通大学,她严谨的办学理念、优良的学风传统以及十六字校训为我打下为人为学的根基,也给了我深入地思考问题的空间,在母校我有10年的求学生涯,她已经成为我成长的第二故乡。

导师是学海中的灯塔,为我的研究生涯指明方向。感谢邹建华导师对我悉心照顾与学业上的帮助。感谢管晓宏院士对我的博士论文提出认真的修改意见,您对论文表达的高屋建瓴令我启发深刻。感谢胡成臣教授对我科研和学习上的指导,您敏锐的思维和富有洞察力的视野给我深刻启迪。

在课题组的这段时间,交到了一辈子的朋友,一同学习也一直向大家学习:张帆、孙秀文、许琛、郑鹏、王瑞龙。

还要感谢在新加坡Xilinx亚太研究院的导师和同事们: Gordon、 Yan、 Henry、 Nguyen,近两年的海外访问实习经历令我的眼界及职业素养有长足进步。

感谢一直关心我的父母,你们的奉献、支持、鼓励和陪伴给了我莫大动力。

博士不只是一个头衔,还是一种自主研究、分析和学习的能力。以开放的心态乐于接收新的知识,碰撞思想的火花。善于发现自己的不足并富有探索精神,才是符合新时代青年的特征,愿为祖国崛起过程添砖加瓦,为祖国繁荣昌盛而奋斗。

人生难得一位伴侣和知己,瑾以本文献给叶乃馨女士。

%\vspace{1em}
%{\color{red} 用于盲审的论文,此页内容全部隐去}。