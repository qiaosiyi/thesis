% !Mode:: "TeX:UTF-8" 


\defaultfont

\BiAppendixChapter{答辩委员会会议决议}{Decision of Defense Committee}

论文围绕网络数据平面的可编程硬件进行了深入研究,选题属于科学前沿,具有重要的理论意义和使用价值。

论文取得的主要创新性成果包括:
\begin{enumerate}
	\item 针对软件定义网络转发节点流表容量受限的问题,提出了一种全局场景下的流表资源可扩展方法,通过与邻居节点建立流表共享机制,提升了全网流表资源动态利用率。
	\item 针对高性能网络转发设备可编程灵活性差的问题,提出了自适应计算的硬件交换系统架构,利用高性能转发芯片以及可编程硬件组成双芯片组的异构结构,提升了数据包处理能力。
	\item 针对当前测量工具性能低以及缺乏统一编程抽象的问题,提出了端侧网络的可编程抽象方法,并设计实现了基于测量技术的网卡硬件系统,将系统使用范围扩展至网络安全、流量控制、拥塞探测等多种场景,相比传统方案,显著提升了系统效率。
\end{enumerate}

论文写作认真、结构合理,表明作者具有坚实宽广的基础理论和系统深入的专门知识,独立从事科研工作能力强。论文工作量饱满,创新性强,是一篇优秀的工学博士学位论文。

答辩委员会投票表决,一致同意通过博士学位论文答辩,并建议授予乔思祎工学博士学位。






